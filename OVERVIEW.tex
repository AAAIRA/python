\section{Overview}
Python adalah bahasa open source yang modern, ditafsirkan, berorientasi objek, yang digunakan dalam semua jenis rekayasa perangkat lunak. Meskipun telah ada selama dua dekade, namun telah mulai digunakan dalam ilmu atmosfer beberapa tahun yang lalu setelah komunitas pengembang berkumpul di atas paket ilmiah standar (misalnya, penanganan array) yang dibutuhkan untuk karya ilmu atmosfer. 
Python sekarang merupakan platform integrasi yang kuat untuk semua jenis ilmu pengetahuan atmosfer, mulai dari analisis data sampai komputasi terdistribusi, dan antarmuka pengguna grafis dengan sistem informasi geografis. Di antara fitur yang menonjol, 
Python memiliki sintaks yang ringkas namun alami untuk kedua array dan nonarrays, membuat program sangat jelas dan mudah dibaca; seperti kata pepatah, "Python adalah pseudocode yang dapat dieksekusi."
Juga, karena bahasa itu ditafsirkan, perkembangannya jauh lebih mudah; Anda tidak perlu menghabiskan waktu ekstra memanipulasi compiler dan linker. Selain itu, struktur data modern dan sifat object-oriented bahasa membuat kode Python lebih kuat dan kurang rapuh. Akhirnya, silsilah open source Python, dibantu oleh basis pengguna dan pengembang besar di industri dan juga sains, 
berarti program Anda dapat memanfaatkan puluhan ribu paket Python yang ada. Ini termasuk visualisasi, perpustakaan numerik, interkoneksi dengan bahasa yang dikompilasi dan lainnya, caching memori, layanan Web, pemrograman antarmuka pengguna grafis mobile dan desktop, 
dan lain-lain. Dalam banyak kasus, beberapa paket ada di masing-masing area domain di atas. Anda tidak terbatas hanya pada apa yang bisa diberikan oleh satu vendor atau bahkan hanya yang bisa diberikan oleh komunitas ilmiah.

Python bahasa script tingkat tinggi, ditafsirkan, interaktif dan berorientasi objek. Python dirancang agar mudah dibaca. Ini menggunakan kata kunci bahasa Inggris sering di mana bahasa lainnya menggunakan tanda baca, dan memiliki konstruksi sintaksis lebih sedikit daripada bahasa lainnya.
Python is interpreted : diproses pada saat runtime oleh interpreter. Anda tidak perlu mengkompilasi program anda sebelum menjalankannya. Ini mirip dengan PERL dan PHP.
Tidak perlu untuk mengkompilasi program anda sebelum mengeksekusi itu. Hal ini merupakan mirip dengan php.
Python is Interactive: Anda dapat benar-benar duduk di prompt Python dan berinteraksi dengan penafsir langsung untuk menulis program Anda.
Python is Object-Oriented: Python mendukung gaya Berorientasi Objek atau teknik pemrograman yang merangkum kode di dalam objek.
Python is a Beginner's Language: Python adalah bahasa yang besar untuk programmer tingkat pemula dan mendukung pengembangan berbagai aplikasi dari pengolahan teks sederhana untuk browser WWW untuk game.
Fitur overview dalam python itu adalah :
\begin {enumerate}
\item Easy-to-learn: Python memiliki beberapa kata kunci, struktur sederhana, dan sintaks yang jelas. Hal ini memungkinkan siswa untuk mengambil bahasa dengan cepat.
\item Easy-to-read: kode Python lebih jelas dan terlihat mata.
\item Easy-to-maintain: kode sumber Python cukup mudah-untuk-menjaga.
\end {enumerate}

Fitur overview terbaik adalah:
\begin {enumerate}
\item IT mendukung metode pemrograman fungsional dan terstruktur serta OOP.
\item Hal ini dapat digunakan sebagai bahasa scripting atau dapat dikompilasi untuk byte-kode untuk membangun aplikasi besar.
\item Ini memberikan tingkat tinggi sangat tipe data dinamis dan mendukung memeriksa jenis dinamis.
\item IT mendukung pengumpulan sampah otomatis.
\item Hal ini dapat dengan mudah diintegrasikan dengan C, C ++, COM, ActiveX, CORBA, dan Java.
\end {enumerate}

Fitur overview dalam python itu adalah:
\begin {enumerate}
\item Easy-to-learn: Python memiliki beberapa kata kunci, struktur sederhana, dan sintaks yang jelas. Hal ini memungkinkan siswa untuk mengambil bahasa dengan cepat.
\item Easy-to-read: kode Python lebih jelas dan terlihat mata.
\item Easy-to-maintain: kode sumber Python cukup mudah-untuk-menjaga.
\end {enumerate}

\subsection{Sejarah Python}
Python dikembangkan oleh Guido van Rossum pada akhir tahun delapan puluhan dan awal tahun sembilan puluhan di National Research Institute for Mathematics and Computer Science di Belanda. Python berasal dari banyak bahasa lain, termasuk ABC, Modula-3, C, C ++, Algol-68, SmallTalk, dan shell Unix dan bahasa script lainnya.
Fitur overview terbaik adalah IT mendukung metode pemrograman fungsional dan terstruktur serta OOP. Hal ini dapat digunakan sebagai bahasa scripting atau dapat dikompilasi untuk byte-kode untuk membangun aplikasi besar. Ini memberikan tingkat tinggi sangat tipe data dinamis dan mendukung memeriksa jenis dinamis. IT mendukung pengumpulan sampah otomatis. Hal ini dapat dengan mudah diintegrasikan dengan C, C ++, COM, ActiveX, CORBA, dan Java. Hal tersebut menjadi terpopuler karena kemudahan bagi programmer yang menjadikan python pemograman terbaik pada tahun 2016.

\subsection{Kekurangan dan Kelebihan Python}
Kelebihan :
\begin{enumerate}
\item Tidak ada tahapan kompilasi dan penyambungan (link) sehingga kecepatan perubahan pada masa pembuatan sistem aplikasi meningkat.
\item Tidak ada deklarasi tipe data yang merumitkan sehingga program menjadi lebih sederhana, singkat, dan fleksible.
\end{enumerate}

\subsection{Paradigma Pemrogramman Python}
Pada program yang selama ini kita buat, kita mendesain program kita berdasarkan fungsi (blok statemen yang memanipulasi data). Hal ini disebut pemrograman procedure-oriented.Ada cara lain untuk mengorganisasi program dengan menggabungkan data dan operasi yang dibungkus dalam suatu obyek yaitu paradigma pemrograman berorientasi obyek.
Dalam perancangan bahasa, para perancang bahasa mengikuti paradigma-paradigma tertentu yang merupakan bentuk pemecahan masalah mengikuti aliran atau “genre” tertentu dari program dan bahasa. Berikut ini merupakan paradigma-paradigma pemograman yang utama:
\begin{enumerate}
\item Imperative programming-> program terdiri dari instruksi yang membentuk perhitungan, menerima input dan menghasilkan output. Contoh bahasa: Fortran, C, dan C++.
\item Object-oriented (OO) programming-> program adalah kumpulan objek yang saling berinteraksi melalui pesan yang mengubah state mereka. Contoh bahasa: Java, C++.
\item Functional programming-> program merupakan kumpulan fungsi matematika dengan input (domain) dan hasil (range). Fungsi-fungsi saling berinteraksi dan berkombinasi mengggunakan komposisi fungsional, kondisional, dan rekursif. Contoh bahasa: Lisp, Scheme,ML
\item Logic (declarative) programming -> memodelkan masalah menggunakan bahasa deklaratif, yang terdiri dari fakta dan aturan. Contoh bahasa : Prolog
\item Event-driven programming-> program merupakan sebuah loop yang secara kontinu  merespon event yang timbul oleh perintah yang tidak terduga.  Event ini berasal dari aksi user pada layar atau sumber lainnya. Contoh bahasa: Visual Basic dan Java.
\item Concurrent programming-> program merupakan sekumpulan proses yang bekerjasama, saling berbagi informasi dari waktu ke waktu tapi biasanya beroperasi secara tidak serempak. Contoh bahasa : SR, Linda, dan HPF.
\end{enumerate}

\subsection{Instalasi Unix dan Linux}
Sistem operasi LINUX adalah salah satu alternatif yang digunakan untuk menggantikan sistem operasi Windows. Perkembangan LINUX sangat cepat karena sistem operasi ini dikembangkan oleh semua orang yang ingin mengembangkannya. Tentu saja membuat sistem operasi LINUX terasa semakin user friendly. Bagi sekolah-sekolah yang sudah mempunyai fasilitas, maka sistem operasi LINUX lebih banyak memiliki kelebihan untuk koneksi ke internet karena LINUX hingga saat ini jarang sekali terkena virus sehingga cukup aman digunakan untuk internet.
Berikut adalah langkah-langkah sederhana untuk menginstal Python di mesin Unix / Linux :
\begin{enumerate}
\item Ikuti link untuk mendownload kode sumber zip yang tersedia untuk Unix / Linux.
\item Download dan ekstrak file.
\item Mengedit Modul / Setup file jika Anda ingin menyesuaikan beberapa pilihan.
\item Jalankan ./configure script33
\item Ini menginstal Python di lokasi standar / usr / local / bin
\item dan pustakanya di / usr / local / lib / pythonXX dimana XX adalah versi Python.
\end{enumerate}

\subsection{Struktur Data}
Python adalah bahasa yang diketik secara dinamis , nilai Python , bukan variabel, tipe carry. Ini berimplikasi pada banyak aspek dari cara fungsi bahasa.
Semua variabel dalam Python menyimpan referensi ke objek, dan referensi ini dilewatkan ke fungsi; sebuah fungsi tidak dapat mengubah nilai referensi variabel dalam fungsi pemanggilannya. Beberapa orang telah memanggil skema parameter-passing ini "Call by object reference."
Di antara bahasa yang diketik secara dinamis, Python cukup tipe-check. Konversi implisit didefinisikan untuk tipe numerik (dan juga boolean), jadi seseorang dapat secara sah mengalikan bilangan kompleks dengan bilangan bulat panjang (misalnya) tanpa pengecoran eksplisit. Namun, tidak ada konversi implisit antara (misalnya) angka dan string; string adalah argumen yang tidak valid terhadap fungsi matematis yang mengharapkan sebuah angka.
