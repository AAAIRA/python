\section{Fitur Overview Pada Python}

\begin{enumerate}
\item Easy-to-learn: Python memiliki beberapa kata kunci, struktur sederhana, dan sintaks yang jelas. $  $Hal ini memungkinkan siswa untuk mengambil bahasa dengan cepat. 
\item Easy-to-read: kode Python lebih jelas dan terlihat mata. 
\item Easy-to-maintain: kode sumber Python cukup mudah-untuk-menjaga.
\item A broad standard library: bulk Python perpustakaan sangat portabel dan cross-platform yang kompatibel pada UNIX, Windows, dan Macintosh. 
\item Interactive Mode: Python memiliki dukungan untuk mode interaktif yang memungkinkan pengujian interaktif dan debugging dari potongan kode. 
\item Portable: Python dapat dijalankan pada berbagai macam platform perangkat keras dan memiliki antarmuka yang sama pada semua platform.
\item Extendable: Anda dapat menambahkan modul tingkat rendah ke interpreter Databases: Python menyediakan antarmuka untuk semua database komersial utama. 
\item GUI Programming: Python mendukung aplikasi GUI yang dapat dibuat dan porting ke banyak panggilan sistem, perpustakaan dan sistem jendela, seperti Windows Scalable: Python menyediakan struktur dan dukungan yang lebih baik untuk program besar dari shell scripting. 
\end{enumerate}

\section{Fitur Overview Terbaik} 
IT mendukung metode pemrograman fungsional dan terstruktur serta OOP. Hal ini dapat digunakan sebagai bahasa scripting atau dapat dikompilasi untuk byte-kode untuk membangun aplikasi besar. Ini memberikan tingkat tinggi sangat tipe data dinamis dan mendukung memeriksa jenis dinamis. IT mendukung pengumpulan sampah otomatis. Hal ini dapat dengan mudah diintegrasikan dengan C, C ++, COM, ActiveX, CORBA, dan Java. Hal tersebut menjadi terpopuler karena kemudahan bagi programmer yang menjadikan python pemograman terbaik pada tahun 2016.

\section{Operasi Interface}

\subsection{Contoh Perhitungan Matematika}


$>$$>$$>$import math 

$>$$>$$>$math.cos(math.pi / 4) 

$>$$>$$>$0.70710678118654757 

$>$$>$$>$math.log(1024, 2) 

$>$$>$$>$10.0 
  
\subsection
{ikhtisar yang sering digunakan oleh programmer}
\begin{enumerate}
	\item pengenalan python
	\item mengisntal python
\item	matematika + numbers
\item	string
	\item list
\item	if elif else
\item	for loop
\item	for else range breake
\item	pass dan continue 
\item	while loop
\item	function
\item	function dan arguments
\item	return velue
\item	lamda function
\item	sope global dan local
\item	more on list
\item	sticks dan queues
	
\end{enumerate}
