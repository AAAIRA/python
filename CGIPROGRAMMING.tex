\section{Common Gateway Interface}
\subsection{Sejarah Common Gateway Interface}
Pada tahun 1993, tim National Center for Supercomputing Applications (NCSA) menulis spesifikasi untuk memanggil executable command line di milis www-talk. [2] [3] [4] Pengembang server Web lainnya menggunakannya, dan telah menjadi standar untuk server Web sejak saat itu. Sebuah kelompok kerja yang dipimpin oleh Ken Coar dimulai pada bulan November 1997 untuk mendapatkan definisi NCSA tentang CGI yang lebih formal. [5] Karya ini menghasilkan RFC 3875, yang menentukan Versi CGI 1.1. Secara khusus disebutkan di RFC adalah kontributor berikut: [6]
\begin{itemize}
\item Rob McCool (penulis Server Web HTTP NCSA)
\item John Franks (penulis GN Web Server)
\item Ari Luotonen (pengembang CERN httpd Web Server)
\item Tony Sanders (penulis Web Server Plexus)
\item George Phillips (pengelola server Web di University of British Columbia)
\end{itemize}
Secara historis skrip CGI sering ditulis menggunakan bahasa C. RFC 3875 Common Gateway Interface (CGI) secara parsial mendefinisikan CGI menggunakan C, [7] seperti dalam mengatakan bahwa variabel lingkungan ``diakses oleh perpustakaan perpustakaan umum getenv () atau variabel lingkungan``. 
\subsection{Common Gateway Interface}
Common Gateway Interface atau disingkat CGI merupakan standar untuk menghubungkan berbagai program aplikasi ke halaman web. CGI mirip dengan program komputer yang menjadi perantara antara standar HTML yang menjadikan tampilan web dengan program lain, seperti basis data (database). Hasil yang diperoleh dari proses pencarian dikirimkan kembali ke halaman web untuk ditampilkan dalam format HTML. CGI (Common Gateway Interface) adalah bentuk dari hubungan interaktif di mana client (browser) bisa mengirimkan suatu masukan kepada server, dan server mengolah masukan tersebut serta mengembalikannya kepada client (browser). Contoh sederhana adalah saat kita menggunakan sebuah mesin pencari. Saat kita menuliskan keyword dan menekan tombol Search maka browser akan mengirimkan keyword tersebut ke server. Keyword tersebut lalu diolah oleh server dan server mengirimkan data hasil pengolahan (yang sesuai dengan keyword yang kita masukkan) ke browser kita. Jadi yang akan kita lihat pada browser adalah  hanya data yang sesuai dengan keyword yang kita masukkan. 
\subsection{Common Gateway Interface}
Common Gateway Interface atau disingkat CGI merupakan standar untuk menghubungkan berbagai program aplikasi ke halaman web. CGI mirip dengan program komputer yang menjadi perantara antara standar HTML yang menjadikan tampilan web dengan program lain, seperti basis data (database). Hasil yang diperoleh dari proses pencarian dikirimkan kembali ke halaman web untuk ditampilkan dalam format HTML. CGI (Common Gateway Interface) adalah bentuk dari hubungan interaktif di mana client (browser) bisa mengirimkan suatu masukan kepada server, dan server mengolah masukan tersebut serta mengembalikannya kepada client (browser). Contoh sederhana adalah saat kita menggunakan sebuah mesin pencari. Saat kita menuliskan keyword dan menekan tombol Search maka browser akan mengirimkan keyword tersebut ke server. Keyword tersebut lalu diolah oleh server dan server mengirimkan data hasil pengolahan (yang sesuai dengan keyword yang kita masukkan) ke browser kita. Jadi yang akan kita lihat pada browser adalah  hanya data yang sesuai dengan keyword yang kita masukkan. 
Untuk dapat menggunakan CGI syarat yang utama adalah server dengan sistem operasi UNIX (beserta variantnya). Namun perlu kita perhatikan bahwa tidak semua server UNIX (gratis) mampu menangani dan melayani CGI. Server-server yang melayani penempatan web yang berlayanan gratis seperti Geocities dan Homepage, tidak akan mengijinkan penggunaan script CGI dalam web kita. Untuk itu kita bisa mencoba Virtual Avenue, Tripod, atau Hypermart.
	Program CGI ditulis dengan menggunakan bahasa yang dapat dimengerti oleh sistem misalnya C/C++, Fortran, Perl, Tcl, Visual Basic, dan lain-lain. Pemilihan bahasa yang digunakan tergantung dari sistem yang digunakan. Jika bahasa pemrograman yang digunakan seperti C atau Fortran maka program-program yang kita buat harus dikompile terlebih dahulu sebelum dijalankan sehingga pada server akan terdapat source code dan program hasil kompilasi. Berbeda jika bahasa yang digunakan yaitu bahasa script seperti PERL, TCL, atau Unix Shell maka hanya akan terdapat script itu sendiri (tanpa ada source code). Jika dibandingkan saat ini banyak orang yang lebih memilih untuk menggunakan sebuah script CGI daripada menggunakan bahasa pemrograman karena lebih mudah untuk di-compile dan dimodifikasi.  
	Pada awalnya CGI merupakansalah satu yang mendekati aplikasi server-side programming.  Program CGI yang paling sering digunakan yaituC++ dan perl.  CGI merupakan bagian dari web server yang dapat berkomunikasi dengan program lain yang ada di server. Dengan CGI web server dapat memanggil program yang dibuat dari berbagai bahasa pemrograman (Common). Interaksi antara pengguna dengan berbagai aplikasi, misalnya database, dapat dijembatani oleh CGI (Gateway). 
	CGI (Common Gateway Interface) merupakan skrip tertua dalam bidang pemrograman web. Skrip bisa didefinisikan sebagai rangkaian dari beberapa instruksi program. Untuk membuat skrip yang dapat dijalankan pada web diperlukan pengetahuan pemograman. 
	CGI sendiri telah muncul sejak teknologi web diperkenalkan di dunia pada awal tahun 1990, bersama dengan kemunculan CERN, web server pertama di dunia. CGI disediakan sebagai tool atau perlengkapan untuk membuat program web. CGI digunakan untuk membuat program-program tampilan web yang lebih interaktif, koneksi ke basis data, bahkan membuat permainan (game). 
	CGI pada masa-masa awalnya dibuat dengan bahasa C, bahasa yang juga digunakan untuk membuat web server pertama yaitu, CERN. CGI kemudian diadopsi oleh NCSA (National Central for Supercomputing Application) web server, dan hingga kini masih digunakan pada Apache Web Server, web server yang paling banyak digunakan oleh komunitas internet saat ini.
	Walaupun demikian CGI bisa juga direalisasikan dengan banyak bahasa pemrograman lain. Mulai dari C, Perl, Phyton, PHP, Tcl/Tk, hingga skrip shell pada UNIX/LINUX. 
	CGI seringkali digunakan sebagai mekanisme untuk mendapatkan informasi dari user melalui fill out form, mengakses basis data (database), atau menghasilkan halaman yang dinamis. meskipun secara prinsip mekanisme CGI tidak memiliki lubang keamana, program atau skrip yang dibuat sebagai CGI dapat memiliki lubang keamanan ataupun tidak sengaja). Potensi lubang keamanan yang digunakan dapat terjadi dengan CGI antara lain: 
\begin{enumerate}
\item Seorang pemakai yang nakal dapat memasang skrip CGI sehingga dapat mengirimkan berkas kata kunci (password) kepada pengunjung yang mengeksekusi CGI tersebut. 
\item Program CGI dipanggil berkali-kali sehingga server menjadi terbebani karena harus menjalankan beberapa program CGI yang menghabiskan memori dan CPU cycle dari web server.
\end{enumerate}
\subsection{Tujuan CGI}
Tujuan dari CGI:
\begin{enumerate}
\item untuk menyediakan akses ke objek dengan overhead komputasi yang lebih sedikit daripada tipikal dari model program CGI.
\item untuk mengurangi biaya overhead sambil memberikan keamanan yang mencakup otentikasi, privasi, dan otorisasi.
\end{enumerate}
\subsection{Tentang CGI}
Sebuah aplikasi web berkomunikasi dengan perangkat lunak client melalui HTTP. HTTP, sebagai protokol yang berbicara menggunakan request dan response menjadikan aplikasi web bergantung kepada siklus ini untuk menghasilkan dokumen yang ingin diakses oleh pengguna. Secara umum, aplikasi web yang akan kita kembangkan harus memiliki satu cara untuk membaca HTTP Request dan mengembalikan HTTP Response ke pengguna. 
	Pada pengembangan web tradisional, kita umumnya menggunakan sebuah web server seperti Apache HTTPD atau nginx sebagai penyalur konten statis seperti HTML, CSS, Javascript, maupun gambar. Untuk menambahkan aplikasi web kita kemudian menggunakan penghubung antar web server dengan program yang dikenal dengan nama CGI (Common Gateway Interface). 
	CGI diimplementasikan pada web server sebagai antarmuka penghubung antara web server dengan program yang akan menghasilkan konten secara dinamis. Program-program CGI biasanya dikembangkan dalam bentuk script, meskipun dapat saja dikembangkan dalam bahasa apapun. Contoh dari bahasa pemrograman dan program yang hidup di dalam CGI adalah PHP.

\subsection{Cara Kerja CGI}
item Web Server yang berhadapan langsung dengan pengguna, menerima HTTP Request dan mengembalikan HTTP Response. 
item Untuk konten statis seperti CSS, Javascript, gambar, maupun HTML web server dapat langsung menyajikannya sebagai HTTP Response kepada pengguna. Konten dinamis seperti program PHP maupun Perl disajikan melalui CGI. CGI Script kemudian menghasilkan HTML atau konten statis lainnya yang akan disajikan sebagai HTTP Response kepada pengguna.
Meskipun terdapat banyak pengembangan selanjutnya dari CGI, ilustrasi sederhana di atas merupakan konsep inti ketika awal pengembangan CGI. Umumnya aplikasi web dengan CGI memiliki kelemahan di mana menjalankan script CGI mengharuskan web server untuk membuat sebuah proses baru. Pembuatan proses baru biasanya akan menggunakan banyak waktu dan memori dibandingkan dengan eksekusi script, dan karena setiap pengguna yang terkoneksi akan mengakibatkan hal ini terhadap server performa aplikasi akan menjadi kurang baik. 
CGI sendiri menyediakan solusi untuk hal tersebut, misalnya FastCGI yang menjalankan aplikasi sebagai bagian dari web server. Bahasa lain juga menyediakan alternatif dari CGI, misalnya Java yang memiliki Servlet. Servlet pada Java merupakan sebuah program yang menambahkan fitur dari server secara langsung. Jadi pada pemrograman dengan Servlet, kita akan memiliki satu web server di dalam program kita, dan pada web server tersebut akan ditambahkan fitur-fitur spesifik aplikasi web kita.
server aplikasi yang aman memiliki antarmuka aman yang menerima permintaan dari browser web. Server juga memiliki program aplikasi untuk mengakses objek, seperti database, dan antarmuka pemrograman aplikasi (API) antara antarmuka aman dan program aplikasi. 
Server aplikasi yang aman dapat berjalan terus menerus sebagai sebuah proses dan dapat mempertahankan informasi keadaan tentang objek. Dalam kasus di mana objek adalah database, database dapat tetap terbuka antara panggilan ke database yang sama, dan server yang aman dapat menyimpan pointer ke rekaman berikutnya untuk dibaca.
Sistem ini membutuhkan komputasi dan overhead memori yang kurang. Server aman terukur di server tambahan tambahan itu dapat ditambahkan dan tugas mereka dapat dibagi sehingga server yang berbeda digunakan untuk tujuan yang berbeda. API antara antarmuka aman dan program aplikasi memungkinkan programmer untuk menggunakan model pemrograman CGI dalam program aplikasi, tanpa memerlukan program pemrogram sesuai dengan antarmuka aman, seperti Distributed Computing Environment (DCE). Fitur dan deskripsi deskripsi lainnya, gambar, dan klaim.
\subsection{Kelebihan CGI} 
Kelebihan yang dimiliki CGI antara lain :
\begin{enumerate}
	\item Skrip CGI dapat ditulis dalam bahasa apa saja, namun barangkali sekitar 90 $  \%  $ program CGI yang ada di tulis dalam Perl 
	\item Protokol CGI yang sederhana
	\item Kefasihan Perl dalam mengolah teks, menjadikan menulis sebuah program CGI cukup mudah dan cepat.
	\item Meski tertua hingga saat ini menurut survey dari Netcraft sekitar 70 $  \%  $ aplikasi di web masih menggunakan CGI. Ini berarti, lebih dari separuh situs Web dinamik yang ada dibangun dengan CGI.
\end{enumerate}

\subsection{Kelemahan CGI} 
Salah satu kelemahannya ialah kecepatan yang rendah. Untuk menghasilkan keluaran program CGI, overhead yang harus ditempuh cukup besar, Dalam kasus CGI Perl, prosesnya sebagai berikut 
\begin{enumerate}
	\item Web server terlebih dahulu akan menciptakan sebuah proses baru dan menjalankan interpreter Perl.
	\item Perl kemudian mengkompilasi script CGI tersebut, baru kemudian menjalankan skrip.
\end{enumerate}
Keseluruhan siklus ini terjadi untuk setiap request. Dengan kata lain, terlalu banyak waktu yang dibuang untuk menciptakan proses dan tidak ada cache skrip yang telah dikompilasi
Namun demikian, mungkin ini tidak lagi menjadi kendala di saat teknologi hardware untuk server sudah sedemikian maju; kecepata prosesor saat ini sudah cukup tinggi. Jika situs web menerima kurang dari sepuluh hingga dua puluh ribu hit CGI per hari, rata-rata mesin web server UNIX yang ada sekarang ini mampu menanganinya dengan baik. 

Dalam kasus CGI Perl, prosesnya sbb: 
\begin{itemize}
	\item Web server terlebih dahulu akan menciptakan sebuah proses baru dan menjalankan interpreter Perl.  
	\item Perl kemudian mengkompilasi script CGI tersebut, baru kemudian menjalankan skrip.\end{itemize}
Keseluruhan siklus ini terjadi untuk setiap request. Dengan kata lain, terlalu banyak waktu dibuang untuk menciptakan proses dan tidak ada cache skrip yang telah dikompilasi. 
Jika sebuah situs web menerima kurang dari sepuluh hingga dua puluh ribu hit CGI per hari, rata-rata mesin web server Unix yang ada sekarang ini mampu menanganinya dengan baik. 
Angka ini relatif, bergantung pada:
\begin{itemize}
	\item Tingkat pembebanan mesin web server untuk melakukan pekerjaan lain (misalnya, mengirim mail dan menjalankan server database)
	\item Aplikasi CGI itu sendiri (sebab beberapa aplikasi CGI berupa skrip tunggal berukuran besar hingga waktu loading-nya cukup lama; umumnya aplikasi CGI yang rumit memecah diri menjadi skrip-skrip terpisah untuk mengurangi waktu loading). \par
	\item Cepat atau lambatnya penampilan halaman web yang diterima klien akan lebih bergantung pada koneksi jaringan.\end{itemize}
\subsection{Penerapan CGI}
Penerapan CGI yang paling umum adalah dalam pemrosesan . Umumnya, form dipergunakan untuk dua kegunaan utama. Yang  sederhana  adalah form yang  dipakai  untuk mengumpulkan informasi dari pengguna dan mengirimkanya ke server. Namun  form juga bisa dipakai untuk keperluan  yang lebih  ``canggih``  seperti timbal balik antara pengguna dan server, misalnya  form  yang memberikan sedaftar pilihan dokumen dalam server kepada pengguna  untuk dipilih. Program CGI di server dibuat untuk mengolah informasi ini  dan kemudian mengirimkan  dokumen – dokumen yang sesuai  dengan  pilihan pengguna.
Contoh nyata penerapan CGI untuk dokumen dinamis ini  misalnya  suatu buku tamu. Pengguna memasukkan informasi seperti nama, alamat, alamat e-mail, dan komentar-komentarnya ke dalam form. Setelah server menerima informasi-informasi tadi, program CGI dapat menyimpanya ke dalam  suatu File atau secara otomatis mengirimkanya lewat e-mailke  suatu  alamat. ProgramCGIjugabisa menampilkan dokumen yang  berisi  informasi  yang
baru saja dikirimkan oleh  pengguna  tadi sembari memberikan  ucapan  terima kasih atas partisipasinya.
Penerapan lain dari CGI adalah sebuah gateway. Artinya adalah  program yangdipergunakan sebagai penghubung  untukmengakses informasi  yang tidak dapat secara langsung dibaca oleh program browser pengguna. Contoh yang nyata adalah gateway yang menghubungkan antara web  server  dengan dengan suatu database server yang besar semacamoracleatau  DB2, yang  memang dapat dilakukan dengan mempergunakan bahasa pemrograman Perl dan DBI extentionta sehingga web server bisamemberikanquery  dalam  SQL (structured query language, yaitu bahasayangdipakai  untuk  melakukan pendefinisian maupun manipulasi terhadapdatabase)ke  server  database Oracle. Setelah informasi dari database keluar, program CGI mengubahnya kedalambentukyangbisa dibaca browser (HTML)  dan  web  server  pada giliranya mengirimkanya kepada browser.
Program CGI pada prinsipnya bisa ditulis dalam bahasa  pemrograman  apa saja,namunkenyataanya tidak  semua  bahasapemrograman cocok  untuk pemrograman CGI. Penerapan CGI dapat sangat kompleks, dan untuk membuat  suatu program CGI menuntut pengetahuan teknis yangcukup  tinggi  akan pemrograman.
\subsection{Keamanan pada CGI} 
CGI dapat menimbulkan lubang keamanan, karena program CGI dapat dijalankan di server lokal dari luar sistem (remote) oleh siapa saja. Apabila program CGI tidak didisain dan dikonfigurasi dengan baik, maka akan terjadi lubang keamanan. Kesalahan yang dapat terjadi antara lain: 
\begin{enumerate}
	\item program CGI mengakses berkas (file) yang seharusnya tidak boleh di akses. Misalnya pernah terjadi kesalahan dalam program phf sehingga digunakan oleh orang untuk mengakses berkas password dari server WW. 
	\item runaway CGI-script, yaitu program berjalan di luar kontrol sehingga mengabiskan CPU cycle dari server WWW.
\end{enumerate}

\subsection{Lubang Keamanan CGI} 
Beberapa contoh lubang keamanan pada CGI 
\begin{enumerate}
	\item CGI dipasang oleh orang yang tidak berhak 
	\item CGI dijalankan berulang-ulang untuk menghabiskan resources (CPU, disk): DoS 
	\item Masalah setuid CGI di sistem UNIX, dimana CGI dijalankan oleh userid web server 
	\item Penyisipan karakter khusus untuk shell expansion 
	\item Kelemahan ASP di sistem Windows 
	\item Guestbook abuse dengan informasi sampah (pornografi) 
	\item Akses ke database melalui perintah SQL (SQL injection).
\end {enumerate}
Untuk menyediakan lebih banyak fungsi keamanan, browser web dan server web dapat menggunakan Distributed Computing Environment (DCE) dari Open Software Foundation (OSF) Cambridge, Mass. Dengan sistem berbasis DCE, permintaan dari browser web disediakan untuk proxy lokal aman (SLP), yang mengarahkan permintaan ke server web DCE-aware via DCE Remote Procedure Call (RPC). Keamanan berdasarkan komunikasi RPC memungkinkan otorisasi selain fitur keamanan lainnya.

\subsection {Web Programming Python}
Python adalah bahasa pemrograman dinamis yang mendukung pemrograman berorientasi obyek. Python dapat digunakan untuk berbagai keperluan pengembangan perangkat lunak dan dapat berjalan di berbagai platform sistem operasi. Seperti halnya bahasa pemrograman dinamis, python seringkali digunakan sebagai bahasa skrip dengan interpreter yang teintergrasi dalam sistem operasi, Para programmer sering menggunakan bahasa python ini untuk membuat sebuah aplikasi baik aplikasi desktop, web, game atau aplikasi yang lainnya. Karena saat ini bahasa python termasuk bahasa yang popular digunakan dikalangan mahasiswa.Python juga dapat dijalankan di hampir semua platform mulai dari GNU/Linux, Windows, dan Machintos. Python termasuk ke dalam general purpose programming language dimana hampir semua tugas pemograman di lingkungan sistem (dalam Linux), jaringan, sampai pemograman berbasis web. Python juga menyediakan framework untuk membuat aplikasi jaringan, Kemampuan python juga dalam mengelola tipe data sangat baik untuk mendeklarasikan suatu variabel dilakukan secara langsung tanpa menyebutkan tipe datanya, ini yang membedakan Python dengan bahasa lain. Python akan menentukan tipe datanya secara otomatis. Python juga mendukung konversi dan perhitungan antar tipe data dengan ketelitian yang tinggi. Saat ini kode python dapat dijalankan pada sistem berbasis}:
\begin{itemize}
	\item Linux/Unix
	\item Windows
	\item Mac OS X 
	\item Java Virtual Machine 
	\item OS/2 
	\item Amiga 
	\item Palm 
	\item Symbian (untuk produk-produk Nokia) \end{itemize}
Pada dasarnya Python merupakan perangkat lunak yang secara default termasuk dalam suatu paket distribusi GNU/Linux. Untuk GNU/Linux distribusi Slackware menggunakan Python versi 2.4, biasanya terdapat pada CD I direktori /slackware/d. Toolkit yang digunakan untuk melakukan instalasi paket di Slackware adalah installpkg, berikut langkah instalasinya: 
\begin{verbatim}
# mount /mnt/cdrom 
# cd /mnt/cdrom/slackware/d 
# installpkg python-2.4.1-i486-1.tgz 
\end{verbatim}
Dari proses instalasi di atas akan membuat beberapa informasi yang tersimpan dalam direktori /usr termasuk di dalamnya berisi library, dokumentasi, file binary, dan informasi lainnya.\par
Kelebihan Python adalah menyediakan modus interaktif yang sangat berguna dalam melakukan latihan dan tes kode. Untuk menulis kode dalam modus interaktif dilakukan dengan memanggil toolkit python pada shell Linux. 
\begin{verbatim}
# python 
Python 2.4.1 (#1, Apr 10 2005, 22:30:36) 
[GCC 3.3.5] on linux2 
Type "help", "copyright", "credits" or "license" 
for more information. 
>>> \end{verbatim}
Tanda “>>>” merupakan suatu prompt dalam modus interaktif Python, selanjutnya Python siap menerima input kode yang dimasukkan. 
Python didistribusikan dengan adanya beberapa lisensi yang berbeda dari beberapa versi. Lihat sejarahnya di Python Copyright. Namun pada prinsipnya Python dapat diperoleh dan dipergunakan secara bebas, bahkan untuk kepentingan komersial. Lisensi Python tidak bertentangan baik menurut definisi sebuah Open Source maupun General Public License (GPL). 
Python merupakan bahasa pemrograman yang mendukung pengembangan aplikasi berbasis desktop dan juga aplikasi berbasis web. Biasanya kalau berhubungan dengan WEB maka orang akan berfikir framework yang digunakan. Tentunya ada beberapa framework yang bisa digunakan untuk membangun aplikasi web berbasis python ini antara lain adalah Django, Web2py, Cherrypy dan lain-lain. Masing-masing framework memiliki aturan khusus dalam penulisan syntax. Framework tersebut mengadopsi struktur yang sama seperti pemrograman CGI. Untuk lebih jelasnya mari kita pelajari pemrograman CGI. 
Common Gateway Interface atau disingkat CGI adalah suatu standar untuk selalu menghubungkan berbagai program aplikasi ke halaman web. CGI mirip sebuah program komputer yang menjadi perantara antara standar HTML yang menjadikan tampilan web dengan program lain, seperti basis data (database). Hasil yang diperoleh dari proses pencarian dikirimkan kembali ke halaman web untuk ditampilkan ke dalam sebuah format HTML.
Python menyediakan modul CGI yang bisa digunakan untuk membuat aplikasi berbasis web. Tentunya python tidak kalah dengan pemrograman berbasis web lain seperti Java, PHP dan lain2. Mari kita lakukan percobaan untuk membuat web dengan menggunakan python. 
Hal Yang paling utama sebelum membuat aplikasi adalah mempersiapkan beberapa komponen aplikasi diantaranya adalah : 
\begin{enumerate}
	\item Menginstal Program Python  
	\item Menginstal Program Web Server Seperti Apache2 atau Xampp  
	\item Setelah kedua program berhasil di install maka langkah selanjutnya adalah mengkonfigurasi file httpd.conf yang berada pada directory web server, pada kesempatan ini saya menggunakan Xampp. 
	\item Buka directory Xampp dan masuk ke folder apache Conf dan cari file httpd dot conf 
	\item Buka file httpd dot conf menggunakan notepad 
	\item Setelah itu simpan 
	\item Selanjutnya kita akan mencoba membuat halaman web dasar pada python 
	\item Buka Notepad dan ketikkan script dbawah ini : 
  \begin{verbatim}
	  \#  !/Python27/python 
	print "Content-type:text/html"
	print 
	print '<html>' 
	print '<head>' 
	print '<title>WEB Python </title>' 
	print '</head>' 
	print '<body>' 
	print '<h1><center>Tutorial Web Programming Python Bagian 1 Python</center></h1>' 
	print 
	print 
	print '<h2><center>Selamat Belajar Bagi Para Pecinta Python</h2></center>' 
	print '</body>' 
	print '</html>' 
  \end{verbatim}
  
	pada script diatas jangan lupa menuliskan posisi directory python.exe ( \$  \#  \$!/Python27/python) 
	setelah itu simpan pada directory xampp folder cgi-bin dengan nama webpy (terserah nama apa saja asalhkan ekstensinya .py) 	
	\item Buka browser dan ketikkan localhost slash cgi-bin slash web dot py pada url dan lihatlah hasilnya
\end{enumerate}

\subsection{Membuat Kamus Menggunakan CGI Python} 
Pertama yang kita butuhkan adalah sebuah kosa kata yang akan digunakan sebagai database, kosa kata tersebut kita convert kedalam format JSON. Untuk prosesnya sebagai berikut. Buatlah sebuah kosa kata bahasa indonesia dan bahasa inggris pada excel dengan header inggris dan indonesia. Jika sudah save as kedalam format .csv lalu di convert ke dalam format .json proses convert  bisa dilakukan secara online disini dan hasilnya akan seperti berikut dan simpan dengan nama kamus .json  
Selanjutnya kita mulai membuat script, buat sebuah file pada folder cgi-bin diserver localhost, tutorial ini menggunakan OS linux, ketikan script berikut. 
\begin{verbatim}
$  \#  $!/usr/bin/python 
import cgi 
import cgitb; cgitb.enable()   
import simplejson as json 
print "Content-type: text/html" 
print 
print """ 
<html> 
<head><title>CGI Script</title></head> 
<body> 
 <h1> Kamus sederhana dengan cgi python</h1> 
 <form method="post" action="index.cgi"> 
 Bahasa Indonesia<br/> 
 <input type="text" name="kata"/></p> 
 <input type="submit" name="submit" value="Terjemahkan"/></p> 
 </form> 
Bahasa Inggris<br/>   
""" 
form = cgi.FieldStorage()  $  \#  $variable form 
cari $  \_  $kata = form.getvalue("kata")  $  \#  $variable mengambil nilai dari input 
location $  \_  $database = open('/home/develop/DW/kamus.json', 'r')  $  \#  $membuka kosa kata bahasa inggris 
bhs $  \_  $inggris = json.load(location $  \_  $database) 

if cari $  \_  $kata:   
 for bhs $  \_  $indonesia in cari $  \_  $kata.split(' '):  
for arti $  \_  $kata in bhs $  \_  $inggris:    
 if arti $  \_  $kata["indonesia"] == bhs $  \_  $indonesia.replace(' ',''): 
 hasil = arti $  \_  $kata['inggris']  
 break 
 else: 
hasil= "arti kata tidak ditemukan"    
  
 print """ 
 <input type="text" name="hasil" value=" $  \%  $s"/> 
 </body> 
 </html> 
 """  $  \%  $ cgi.escape(hasil) 
\end{verbatim}
Jika sudah save dengan nama kamus.cgi sebagai contoh dan buka browser ketikan pada url http//localhost/cgi-bin/kamus.cgi jika muncul form input coba di tester ketikan nama kata dalam bahasa indonesia.
